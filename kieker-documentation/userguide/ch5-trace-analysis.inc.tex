%%%%%%%%%%%%%%%%%%%%%%%%%%%%%%%%%%%%%%%%%
% Trace Analysis and Monitoring
%
% $Date$
% $Rev$:
% $Author$

\chapter{\KiekerTraceAnalysis{} Tool}\label{chap:aspectJ}

\KiekerTraceAnalysis{} implements the special feature of \Kieker{} allowing to %
monitor, analyze, and visualize (distributed) traces of method executions and %
corresponding timing information. For this purpose, it includes monitoring probes employing %
AspectJ~\cite{AspectJ-WebSite}, Java~EE Servlet~\cite{JavaServletTechnology-WebSite}, %
Spring~\cite{Spring-WebSite}, and Apache~CXF~\cite{CXF-WebSite} technology. %
Moreover, it allows to reconstruct and visualize architectural models of the %
monitored systems, e.g., as sequence and dependency diagrams. %

Section~\ref{chap:example} already introduced parts of the monitoring record %
type \class{OperationExecutionRecord}. \KiekerTraceAnalysis{} uses this record %
type to represent monitored executions and associated trace and session information. %
Figure~\ref{fig:OperationExecutionRecordClassDiagramComplete} shows a class diagram %
with all attributes of the record type \class{OperationExecutionRecord}. %
The attributes \method{className}, \method{operationName}, %
\method{tin}, and \method{tout} have been introduced before. %
The attributes \method{traceId} and \method{sessionId} are used to store %
trace and session information; \method{eoi} and \method{ess} contain control-flow %
information needed to reconstruct traces from monitoring data. %
For details on this, please refer to our technical %
report~\cite{vanHoornRohrHasselbringWallerEhlersFreyKieselhorst2009TRContinuousMonitoringOfSoftwareServicesDesignAndApplicationOfTheKiekerFramework}.

\begin{figure}[hb]\centering
\includegraphics[scale=0.8]{images/kieker_OperationExecutionRecord-complete-modified}%
\caption{The class diagram of the operation execution record}
\label{fig:OperationExecutionRecordClassDiagramComplete}
\end{figure}

\enlargethispage{1cm}

\noindent Section~\ref{sec:traceMonitoring} describes how to instrument Java %
applications for monitoring trace information. %
It presents the technology-specific probes provided by \Kieker{} for this %
purpose---with a focus on AspectJ. %
Additional technology-specific probes can be implemented based on the existing %
probes. %
Section~\ref{sec:traceAnalysisTool} presents the %
tool which can be used to analyze and visualize the recorded trace %
data.  Examples for the available analysis and visualization outputs %
provided by \KiekerTraceAnalysis{} are presented in %
Section~\ref{sec:traceAnalysisExamples}.

\section{Monitoring Trace Information}\label{sec:traceMonitoring}

The following Sections describe how to use the monitoring probes based on %
AspectJ (Section~\ref{sec:traceAnalysis:instr:AspectJ}), %
the Java Servlet~API (Section~\ref{sec:traceAnalysis:instr:servlet}), %
the Spring Framework (Section~\ref{sec:traceAnalysis:instr:spring}), and %
Apache~CXF (Section~\ref{sec:traceAnalysis:instr:cxf}) provided %
by \Kieker{}. %

\subsecion{AspectJ-based Instrumentation}

\paragraph{NOTE} This part is illustrated in getting started. 
However, a complete trace tutorial with a better example should be setup.

\subsection{Servlet Filters}\label{sec:traceAnalysis:instr:servlet}

The Java Servlet API~\cite{JavaServletTechnology-WebSite} includes the %
\class{javax.servlet.Filter} interface. It can be used to implement %
interceptors for incoming HTTP requests. %
\Kieker{} includes the probe %
\class{SessionAndTraceRegistrationFilter} which implements the %
\class{javax.servlet.Filter} interface. %
It initializes the session and trace information for incoming requests. %
If desired, it additionally creates an \class{OperationExecutionRecord} for each %
invocation of the filter and passes it to the \class{MonitoringController}.

% \enlargethispage{1.5cm}

Listing~\ref{lst:OperationExecutionRegistrationAndLoggingFilterInWebXML} %
demonstrates how to integrate the \class{SessionAndTraceRegistrationFilter} %
in the \file{web.xml} file of a web application.

The Java~EE Servlet container example described in Appendix~\ref{appendix:JavaEEServletExample} employs the %
\class{SessionAndTraceRegistrationFilter}.

\pagebreak

\setXMLListing
\lstinputlisting[firstline=50,lastline=61,firstnumber=50,%
caption=\class{SessionAndTraceRegistrationFilter} in a \file{web.xml} file,%
label=lst:OperationExecutionRegistrationAndLoggingFilterInWebXML]%
{\JavaEEServletExampleDir/jetty/webapps/jpetstore/WEB-INF/web.xml}


\subsection{Spring}\label{sec:traceAnalysis:instr:spring}

The Spring framework~\cite{Spring-WebSite} provides interfaces for intercepting %
Spring services and web requests. %
\Kieker{} includes the probes %
\class{OperationExecutionMethodInvocationInterceptor} and
\class{OperationExecutionWebRequestRegistrationInterceptor}. %
The \class{OperationExecutionMethodInvocationInterceptor} is similar to the %
AspectJ-based probes described in the previous section and monitors method %
executions as well as corresponding trace and session information. %
The \class{OperationExecutionWebRequestRegistrationInterceptor} intercepts %
incoming Web requests and initializes the trace and session data for this %
trace. If you are not using the \class{OperationExecutionWebRequestRegistrationInterceptor}, %
you should use one of the previously described Servlet filters to register %
session information for incoming requests %
(Section~\ref{sec:traceAnalysis:instr:servlet}).

See the Spring documentation for instructions how to add the interceptors %
to the server configuration.

\subsection{CXF SOAP Interceptors}\label{sec:traceAnalysis:instr:cxf}

The Apache~CXF framework~\cite{CXF-WebSite} allows to implement interceptors for web service calls, %
for example, based on the SOAP web service protocol. %
\Kieker{} includes the probes %
\class{OperationExecutionSOAPRequestOutInterceptor}, %
\class{OperationExecutionSOAPRequestInInterceptor}, %
\class{OperationExecutionSOAPResponseOutInterceptor}, and %
\class{OperationExecutionSOAPResponseInInterceptor} which can be used to %
monitor SOAP-based web service calls. %
Session and trace information is written to and read from the SOAP header of %
service requests and responses allowing to monitor distributed traces. %
See the CXF documentation for instructions how to add the interceptors %
to the server configuration.

\pagebreak

\section{Trace Analysis and Visualization}\label{sec:traceAnalysisTool}

\enlargethispage{0.5cm}

Monitoring data including trace information can be analyzed and visualized with the \KiekerTraceAnalysis{} tool which is included in the \Kieker{} binary as well.\\

\WARNBOX{
In order to use this tool, it is necessary to install two third-party programs:
\begin{enumerate}
\item \textbf{Graphviz} A graph visualization software which can be downloaded from \url{http://www.graphviz.org/}.
\item \textbf{GNU PlotUtils} A set of tools for generating 2D plot graphics which can be downloaded from \url{http://www.gnu.org/software/plotutils/} (for Linux) and from \url{http://gnuwin32.sourceforge.net/packages/plotutils.htm} (for Windows).
\item \textbf{ps2pdf} The \file{ps2pdf} tool is used to convert ps files to pdf files.
\end{enumerate}
Under Windows it is recommended to add the \dir{bin/} directories of both tools to the ``path'' environment variable. It is also possible that the GNU PlotUtils are unable to process sequence diagrams. In this case it is recommended to use the Cygwin port of PlotUtils.
}

\vspace{1mm}

\noindent Once both programs have been installed, the \KiekerTraceAnalysis{} tool can be used. It can be accessed via the wrapper-script \file{trace-analysis.sh} or \file{trace-analysis.bat} (Windows) in the \dir{bin/} directory. Non-parameterized calls of the scripts print all possible options on the screen. %, as listed in Appendix~\ref{appendix:wrapperScripts:traceAnalysis}.

The commands shown in Listings~\ref{lst:traceAnalysis:sequenceDiagram} and \ref{lst:traceAnalysis:sequenceDiagramWin} generate a sequence diagram as well as a call tree to an existing directory named \dir{out/}. The monitoring data is assumed to be located in the directory \dir{/tmp/kieker-20110428-142829399-UTC-Kaapstad-KIEKER/} or \dir{\%temp\%$\backslash{}$kieker-20100813-121041532-UTC-virus-KIEKER} under Windows. %

\setBashListing
\input{ch5-trace-analysis_Produce_Sequence_Diagram.inc}
\input{ch5-trace-analysis_Produce_Sequence_Diagram-win.inc}

\enlargethispage{1cm}

\WARNBOX{%
The Windows \file{.bat} wrapper scripts (including \file{trace-analysis.bat}) must be executed from within %
the \dir{bin/} directory.
}

\pagebreak

The resulting contents of the \dir{out/} directory should be similar to %
the following tree:

\begin{figure}[H]
\begin{graybox}
\dirtree{%
.1 \DirInDirTree{out/}.
.2 deploymentSequenceDiagram-6120391893596504065.pic.
.2 callTree-6120391893596504065.dot.
.2 system-entities.html.
}
\end{graybox}
\end{figure}

\noindent The \file{.pic} and \file{.dot} files can be converted into other formats, %
such as \file{.pdf}, by using the \textit{Graphviz} and \textit{PlotUtils} tools %
\file{dot} and \file{pic2plot}. %
The following Listing~\ref{lst:traceAnalysis:convertDiagrams} demonstrates this. %

% The generated diagrams are shown in the following %
% Figures~\ref{fig:traceAnalysis:callTree} and~\ref{fig:traceAnalysis:allocSeqDiagr}.

\input{ch5-trace-analysis_Convert_Diagrams.inc}

% \begin{figure}[H]\centering
%   \subfigure[]{\label{fig:traceAnalysis:callTree}%
%   \includegraphics[height=0.4\textheight]{images/callTree}
%   }%
%   \subfigure[]{\label{fig:traceAnalysis:allocSeqDiagr}%
%   \includegraphics[height=0.4\textheight]{images/allocationSequenceDiagram}
%   }%
%
%   \caption{Call Tree~\subref{fig:traceAnalysis:callTree} and Allocation Sequence Diagram~\subref{fig:traceAnalysis:allocSeqDiagr}}
% \end{figure}

\NOTIFYBOX{The scripts \file{dotPic-fileConverter.sh} and \file{dotPic-fileConverter.bat} %
convert all \file{.pic} and \file{.dot} in a specified directory. %
%See Appendix~\ref{appendix:wrapperScripts:dotPicFileConverter} for details.
}

\vspace{5mm}

Examples of all available visualization are presented in the following %
Section~\ref{sec:traceAnalysisExamples}.

\pagebreak

\section{Example \KiekerTraceAnalysis{} Outputs}\label{sec:traceAnalysisExamples}
\input{ch5-trace-analysis-exampleOutputs.inc}
