%%%%%%%%%%%%%%%%%%%%%%%%%%%%%%%%%%%%%%%%%
% Quick Start Example
%
% $Date$
% $Rev$:
% $Author$


\chapter{Quick Start Example}\label{chap:example}

%%
\section{Analysis with \KiekerAnalysisPart{}}\label{sec:example:analysis}

\paragraph{NOTE} This section is outdated.

%The last task of a successful application monitoring is the analysis of the collected information.
In this section, the monitoring data recorded in the previous section is %
analyzed with \KiekerAnalysisPart{}. %
% The second step is to write a suitable analyzer. And finally the analyzer is used to aggregate the information in a sensible way.
For this quick example guide, the analysis tool is very simple and does not show %
the full potential of \Kieker{}. For more detail, read %
Chapter~\ref{chap:componentsAnalysis} to learn which plugins, i.e., readers %
and filters, are included in \Kieker{}, how to use them, and how to develop %
custom plugins. %
Chapter~\ref{chap:aspectJ} presents the \KiekerTraceAnalysis{} tool, which %
is also based on \KiekerAnalysisPart{}.
\KiekerAnalysisPart{} has a dependency to the Eclipse Modeling Framework~(EMF).%
\footnote{\url{http://www.eclipse.org/modeling/emf/}} %
For this reason, we are using the \file{\mainJarEMF{}} that is a variant of %
the \file{\mainJar{}}, additionally including the required EMF dependencies. %
When using the \file{\mainJar{}}, the \file{org.eclipse.emf.\-*.jar} files (to %
be found in \Kieker's \file{lib/} directory) need to be added to the classpath. %

\begin{figure}[H]
\begin{graybox}
\dirtree{%
.1 \DirInDirTree{examples/}. %\DTcomment{The root directory of the project}.
.2 \DirInDirTree{userguide/}.
.3 \DirInDirTree{ch2--manual-instrumentation/}.
.4 \DirInDirTree{build/}\DTcomment{Directory for the Java class files}.
.4 \newFileDirInDirTree{\DirInDirTree{lib/}}\DTcomment{Directory for the required libraries}.
.5 \newFileDirInDirTree{\mainJarEMF}.
.4 \DirInDirTree{src/}\DTcomment{The directory for the source code files}.
.5 \DirInDirTree{.../manual/}.
.6 \ldots.
.6 \newFileDirInDirTree{BookstoreAnalysisStarter.java}.
}
\end{graybox}
\caption{Directory layout of the example application with the analysis files highlighted}
\label{lst:analysisExampleLayout}
\end{figure}

\noindent The analysis application developed in this section comprises the file %
\file{BookstoreAnalysisStarter.java}, as shown in Figure~\ref{lst:analysisExampleLayout}. %
This file can also be found in the directory \dir{\manualInstrumentedBookstoreApplicationReleaseDirDistro{}/}.
The file sets up the basic pipe-and-filter configuration depicted in Figure~\ref{fig:example:ch2:pipe-and-filter}: %
\Kieker{}'s file system reader (\class{FSReader}) reads monitoring records %
from a file system monitoring log (as produced in the previous Section~\ref{sec:example:monitoring}) %
and passes these to the \class{TeeFilter} plugin; the \class{TeeFilter} plugin %
reads events of arbitrary type (i.e., Java \class{Object}), prints them to a %
configured output stream, and also relays them to filters connected to the %
filter's output port \method{relayedEvents}. %

\begin{figure}[h]
\includegraphics[width=\textwidth]{images/ch2-example-pnp}
\caption{Example pipe-and-filter configuration}
\label{fig:example:ch2:pipe-and-filter}
\end{figure}

\enlargethispage{0.5cm}

\KiekerAnalysisPart{} pipe-and-filter configurations can %
be created programmatically, i.e., by configuring, instantiating, and %
connecting the plugins in a Java program.%
\footnote{As an alternative, a web-based user interface is available for \Kieker{} \cite{KiekerWebSite}} %
For the example, this is demonstrated in Listing~\ref{lst:BookstoreAnalysisStarter}, %
which shows an excerpt from the \class{BookstoreAnalysisStarter}'s \method{main} %
method. %

% \pagebreak

\setJavaCodeListing
\lstinputlisting[gobble=4,caption=BookstoreAnalysisStarter.java (excerpt from \method{main} method),label=lst:BookstoreAnalysisStarter,firstline=37,firstnumber=37,lastline=56]%
{\manualInstrumentedBookstoreApplicationDir/src/kieker/examples/userguide/ch2bookstore/manual/BookstoreAnalysisStarter.java}

% \pagebreak

\noindent The \class{BookstoreAnalysisStarter} follows a simple scheme. Each %
analysis tool has to create at least one \class{AnalysisController} which can be %
seen in Listing~\ref{lst:BookstoreAnalysisStarter} in line~38. Then, the plugins, %
which may be readers or filters, are configured, and instantiated. The usage of the  %
constructor ensures that the component is registered with the analysis instance. %
Lines~41--43 configure, instantiate, and register the file system monitoring %
log reader, which uses the command-line argument value as the input directory. %
The application expects the %
output directory from the earlier monitoring run (see Section~\ref{sec:example:monitoring}) %
as the only argument value, which must be passed manually. %
Lines~46--49 configure, instantiate, and register the \class{TeeFilter}, %
which outputs received events to the standard output. Lines~52 and~53 connect %
the \class{TeeFilter}'s input port to the filesystem reader's output port. %
The analysis is started by calling its \method{run} method (line~56). %


The Listings~\ref{lst:bookstoreAnalysisStarterLinux} and \ref{lst:bookstoreAnalysisStarterWin} %
describe how the analysis application can be compiled and executed under \UnixLikeSystems{} and Windows.

\setBashListing
\enlargethispage{1.0cm}
\input{ch2-quickstart-example_Compile_Run_Example_2.inc.tex}

\input{ch2-quickstart-example_Compile_Run_Example_2.inc-win.tex}

\noindent You need to make sure that the application gets the correct path from the monitoring run.
The \class{TeeFilter} prints an output message for each record received. %
An example output can be found in Appendix~\ref{sec:appendix:manualInstrumentation:output}.
% A possible display of the run can be found in the appendix of this tutorial.
